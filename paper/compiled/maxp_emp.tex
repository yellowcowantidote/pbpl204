\documentclass[11pt,article,oneside]{memoir}
\usepackage{paper/.pandoc/org-preamble-xelatex}
\usepackage{lscape}
\usepackage{flafter}

\usepackage{longtable}

\newlength{\cslhangindent}
\setlength{\cslhangindent}{1.5em}
\newenvironment{cslreferences}%
  {\setlength{\parindent}{0pt}%
  \everypar{\setlength{\hangindent}{\cslhangindent} \setlength{\parskip}{6pt}}\ignorespaces}%
  {\par}


\title{\bigskip A Computational Approach for Defining Metropolitan
Employment Centers: Polycentric Urban Form \& the p-Regions Problem}


\author{\large Elijah
Knaap\vspace{0.05in} \newline\normalsize\emph{University of
California-Riverside} \newline\footnotesize \url{knaap@ucr.edu}\vspace*{0.2in}\newline  \and \large Ran
Wei\vspace{0.05in} \newline\normalsize\emph{University of
California-Riverside} \newline\footnotesize \url{ranwei@ucr.edu}\vspace*{0.2in}\newline  \and \large Sergio
Rey\vspace{0.05in} \newline\normalsize\emph{University of
California-Riverside} \newline\footnotesize \url{serge.rey@ucr.edu}\vspace*{0.2in}\newline }


\date{}


\begin{document}
\setkeys{Gin}{width=1\textwidth}
\setromanfont[Mapping=tex-text]{Spectral}
\setsansfont[Mapping=tex-text]{Roboto}
\setmonofont[Mapping=tex-text,Scale=0.8]{Oxygen Mono}
\chapterstyle{article-4}
\pagestyle{kjh}

\published{January 2020.}

\maketitle



\begin{abstract}\vspace*{-2cm}


\noindent Recent work has suggested a departure from the familiar
monocentric city model popularized into a polycentric urban form defined
by multiple employment centers that form along regional transportation
corridors. Importantly, these studies have demonstrated the ways in
which the forces of localization and agglomeration can work together to
define a pallette of well-defined employment centers that may specialize
in certain industries but nonetheless increase the economic performance
of firms that choose to locate inside Despite the important advances in
polycentric research, we know little about the properties of these
employment centers across the range of metropolitan regions, since the
existing research has been constrained to a handful of large
metropolitan regions. Thus, in this paper, we employ a novel
modification of the max-p algorithm by extending it to incorporate two
threshold constraints (satisfying both total employment and density
thereof) and applying our method to every metropolitan region in the
United States. By leveraging the max-p algorithm we uncover the
polycentric urban form in each of America's metropolitan regions and
describe the size, shape, and composition of the resulting employment
centers. Using these results, we describe how the identified employment
centers can be leveraged as the basis of economic development,
sustainability, and equitable transportation planning.

\end{abstract}


\hypertarget{introduction}{%
\section{Introduction}\label{introduction}}

Over the last several decades, American metropolitan regions have borne
witness to major economic restructuring as the American economy has
shifted from manufacturing and XXX to a service and knowledge-based
economy. Concurrent with these trends is a less severe but equally
important restructuring of the \emph{spatial structure} of American
employment, as certain industries have begin gravitating toward more
suburban locations to be nearer to their employment bases, whereas other
industries are flocking to inner cities. These trends suggest a critical
reckoning for the study of agglomeration economies that provides the
foundation of modern urban economics and regional science. Recent work
has suggested a departure from the familiar monocentric city model
popularized by Muth (1969) into a polycentric urban form defined by
multiple employment centers that form along regional transportation
corridors (Knaap et al. 2016; Giuliano and Small 1991). Importantly,
these studies have demonstrated the ways in which the forces of
localization and agglomeration can work together to define a pallette of
well-defined employment centers that may specialize in certain
industries but nonetheless increase the economic performance of firms
that choose to locate inside. Furthermore, more recent evidence has
demonstrated that these employment centers also provide a range of
benefits for achieving global sustainability such as reduced
transportation costs and lower trip durations.

Despite the important advances in polycentric research, we know little
about the properties of these employment centers across the range of
metropolitan regions, since the existing research has been constrained
to a handful of large metropolitan regions. Existing work often adopts a
relatively simple definition of an employment center: a set of
contiguous spatial units whose combined total employment and employment
density meet some pre-defined thresholds. But this process of
identifying employment centers is prohibitively laborious, since it
requires the tedious and iterative process of aggregating spatial units,
testing whether each constraint is met, then re-aggregating and testing
again. Luckily, over the last decade, a family of ``regionalization''
algorithms has grown from the tradition of spatial optimization which
excels at identifying the maximum number of \emph{p}-distinct regions
(Duque, Church, and Middleton 2011) in a spatial dataset. Thus, in this
paper, we employ a novel modification of the max-p algorithm by
extending it to incorporate two threshold constraints (satisfying both
total employment and density thereof) and applying our method to every
metropolitan region in the United States (Duque, Anselin, and Rey 2012)

As a result, we identify a set of American employment centers at a
national scale. By leveraging the max-p algorithm we uncover the
polycentric urban form in each of America's metropolitan regions and
describe the size, shape, and composition of the resulting employment
centers. Using these results, we describe how the identified employment
centers can be leveraged as the basis of economic development,
sustainability, and equitable transportation planning.

\hypertarget{spatial-structure-as-a-vehicle-for-urban-policy}{%
\section{Spatial Structure as a Vehicle for Urban
Policy}\label{spatial-structure-as-a-vehicle-for-urban-policy}}

Large portions of urban planning, regional science, and economic
geography research are devoted to understanding how spatial patterns
like development intensity and the separation of land uses influence
social and economic activities. Part of this work is focused on patterns
like urban sprawl, and the effect they have on GHG emissions, housing
prices, and social welfare. Another--much older--portion of this work
focuses on measuring form to understand the optimal location of
resources to maximize economic output

Regardless of the perspective, the literature from across the many
disciplines makes clear that urban policy measures are often designed to
reshape urban form explicitly, or to make the most of existing form to
capitalize on the social, economic, and environmental benefits.

\hypertarget{the-monocentric-model}{%
\subsection{The Monocentric Model}\label{the-monocentric-model}}

von thunen, muth, mills, brueckner; agglomeration as a foundation of
urban economic and economic development policy.

public transportation and transportation efficiency more generally

\hypertarget{polycentric-urban-development}{%
\subsection{Polycentric Urban
Development}\label{polycentric-urban-development}}

More recently, especially in newer cities like LA, polycentrism rather
than monocentrism is the norm. Giuliano \& small show this works well
for economic development. On the other hand public transportation is
notoriously difficult in LA because its so spread out. (More rencnt
eamples of it working??)

In DC, and Baltimore, polycentrism seems to work for both economic
development \emph{and} transportation, with cascading environmental
benefits as a result Knaap et al. (2016)

So polycentrism is not only an important and useful model for
encouraging both growth and sustainability, but it also may be emerging
as the dominant urban form in the US and across the world.

\hypertarget{urban-spatial-structure-as-a-p-regions-problem}{%
\section{\texorpdfstring{Urban Spatial Structure as a \emph{p}-Regions
problem}{Urban Spatial Structure as a p-Regions problem}}\label{urban-spatial-structure-as-a-p-regions-problem}}

the p-regions problem Duque, Church, and Middleton (2011)

max-p is a particular version of the problem Duque, Anselin, and Rey
(2012)

recently a number of extensions to max-p have been explored She, Duque,
and Ye (2017) Duque, Vélez-Gallego, and Echeverri (2018)

Li, Church, and Goodchild (2014) the p-compact-regions problem for urban
economic modeling

Here, we propose a novel extension of the max-p regions \emph{idea}. In
this case, the modification is

\hypertarget{modifying-max-p-to-identify-urban-polycenters}{%
\section{Modifying Max-p to Identify Urban
Polycenters}\label{modifying-max-p-to-identify-urban-polycenters}}

In the original formulation of the p-regions problem, a study area is
comprised of a set of polygons that must be aggregated to form the
largest number of aggregate regions, subject to certain constraints.
Here, every unit in the study area should be assigned to a particular
region until the study space is partitioned fully. In the case of
identifying employment centers, however, the original p-regions
formulation requires two important extensions. First it is not
appropriate to partition the entire study space

In the classic max-p problem, regions are identified by (1) satisfying a
threshold constraint, and (2) optimizing some objective function, which
is typically multivariate similarity among the enclave's attributes.
When identifying employment centers, however, this objective function
assumes the form of a second constraint; here the alogirthm must satisfy
the total employment constraint within the region, but also must satisfy
an employment density criterion.

If we take employment density as the objective to optimize

\hypertarget{polycentric-urban-form-in-american-metropolitan-regions}{%
\section{Polycentric Urban Form in American Metropolitan
Regions}\label{polycentric-urban-form-in-american-metropolitan-regions}}

\hypertarget{quantity-shape-and-configuration}{%
\subsection{Quantity, Shape, and
Configuration}\label{quantity-shape-and-configuration}}

\begin{itemize}
\item
  radial centers around a central blob (i.e.~new centers along
  transportation corridors extending from an original mono center a la
  baltimore)?

  \begin{itemize}
  \tightlist
  \item
    long and extended along transpo lines? (like I-270)
  \item
    round and compact in walkable dense nodes? (like bethesda)
  \end{itemize}
\item
  distinct subnodes without an original center (a la LA?)
\item
  relationship between city size, population size, or econommy size and
  number of centers?
\end{itemize}

\hypertarget{composition}{%
\subsection{Composition}\label{composition}}

\hypertarget{evolution-over-time}{%
\subsection{Evolution over time ???}\label{evolution-over-time}}

(this would be really cool but would extend the scope a lot)

\hypertarget{discussion-policy-implications}{%
\section{Discussion \& Policy
Implications}\label{discussion-policy-implications}}

\hypertarget{conclusion}{%
\section{Conclusion}\label{conclusion}}

\hypertarget{references}{%
\section*{References}\label{references}}
\addcontentsline{toc}{section}{References}

\hypertarget{refs}{}
\begin{cslreferences}
\leavevmode\hypertarget{ref-duque_max-p-regions_2012}{}%
Duque, Juan C., Luc Anselin, and Sergio J. Rey. 2012. ``The
max-p-regions problem.'' \emph{Journal of Regional Science} 53 (3):
397--419. \url{https://doi.org/10.1111/j.1467-9787.2011.00743.x}.

\leavevmode\hypertarget{ref-Duque2017}{}%
Duque, Juan Carlos, Mario C. Vélez-Gallego, and Laura Catalina
Echeverri. 2018. ``On the Performance of the Subtour Elimination
Constraints Approach for the p -Regions Problem: A Computational
Study.'' \emph{Geographical Analysis} 50 (1): 32--52.
\url{https://doi.org/10.1111/gean.12132}.

\leavevmode\hypertarget{ref-Duque2011a}{}%
Duque, Juan C., Richard L. Church, and Richard S. Middleton. 2011. ``The
p-Regions Problem.'' \emph{Geographical Analysis} 43 (1): 104--26.
\url{https://doi.org/10.1111/j.1538-4632.2010.00810.x}.

\leavevmode\hypertarget{ref-Giuliano1991}{}%
Giuliano, Genevieve, and Kenneth A Small. 1991. ``Subcenters in the Los
Angeles region.'' \emph{Regional Science and Urban Economics} 21 (2):
163--82. \url{https://doi.org/10.1016/0166-0462(91)90032-I}.

\leavevmode\hypertarget{ref-Knaap2016}{}%
Knaap, Elijah, Chengri Ding, Yi Niu, and Sabyasachee Mishra. 2016.
``Polycentrism as a sustainable development strategy: empirical analysis
from the state of Maryland.'' \emph{Journal of Urbanism: International
Research on Placemaking and Urban Sustainability} 9 (1): 73--92.
\url{https://doi.org/10.1080/17549175.2015.1029509}.

\leavevmode\hypertarget{ref-Li2014}{}%
Li, Wenwen, Richard L. Church, and Michael F. Goodchild. 2014. ``An
extendable heuristic framework to solve the p-compact-regions problem
for urban economic modeling.'' \emph{Computers, Environment and Urban
Systems} 43 (January): 1--13.
\url{https://doi.org/10.1016/j.compenvurbsys.2013.10.002}.

\leavevmode\hypertarget{ref-Muth1969}{}%
Muth, Richard F. 1969. ``Cities and housing; the spatial pattern of
urban residential land use.'' \emph{American Journal of Agricultural
Economics} 91 (1): 19--41.
\url{http://trid.trb.org/view.aspx?id=545388}.

\leavevmode\hypertarget{ref-She2017}{}%
She, Bing, Juan C. Duque, and Xinyue Ye. 2017. ``The
Network-Max-P-Regions model.'' \emph{International Journal of
Geographical Information Science} 31 (5): 962--81.
\url{https://doi.org/10.1080/13658816.2016.1252987}.
\end{cslreferences}


\end{document}
